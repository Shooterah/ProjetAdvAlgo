
\documentclass[12pt]{article}
\AddToHook{cmd/section/before}{\clearpage}
\usepackage{graphicx}
\usepackage{amsmath}
\usepackage{amssymb}
\usepackage{amsthm}
\usepackage{amscd}
\usepackage{amsfonts}
\usepackage{amstext}
\usepackage{hyperref}

\usepackage{listings}
\usepackage{color}

\definecolor{dkgreen}{rgb}{0,0.6,0}
\definecolor{gray}{rgb}{0.5,0.5,0.5}
\definecolor{mauve}{rgb}{0.58,0,0.82}

\lstset{frame=tb,
  language=Python,
  aboveskip=3mm,
  belowskip=3mm,
  showstringspaces=false,
  columns=flexible,
  basicstyle={\small\ttfamily},
  numbers=none,
  numberstyle=\tiny\color{gray},
  keywordstyle=\color{blue},
  commentstyle=\color{dkgreen},
  stringstyle=\color{mauve},
  breaklines=true,
  breakatwhitespace=true,
  tabsize=3
}


\begin{document}

\title{Advanced Programming Report - Knapsack Problem}
\author{Florian Doffemont - Jeoffrey Pereira - Jocelyn Hauf - Arthur Micol}
\date{}

\maketitle

\tableofcontents

\section{Introduction}

    \subsection{About This Report}
        This report is going to talk about, describe the Advanced Programming project that we have been given to do in groups of 4. The subject of the project is to implement the Knapsack Problem with different types of algorithms. The document will have as section the different types of algorithms that will be explained, shown, analyzed and compared with the others. Some sections will also have treated other problems (Multidimensional...). All external sources like datasets will be quoted in the last section (Sources)
    
    \subsection{Knapsack Problem}
        The Knapsack problem consists in choosing the best combination of objects (which have a weight and a value) in order to have a maximum value in the backpack (which has a maximum weight not to exceed). It is a NP complete problem.
    
    \subsection{Multidimensional Knapsack Problem}
        Here, the goal is the same as the simple Knapsack problem, with the difference that we have to manage not only one dimension of the backpack (the weight) but several dimensions (weight, volume, etc..), we can see it as a vector with n dimensions. So an object will have a value and n dimensions. It will be necessary to have the best combination of object to have the best value in the backpack without that the objects "overflow" of all the dimensions of the backpack. Of course, the "size" of all dimensions are predefined and should not be exceeded.
    
    \subsection{Multiple Knapsack Problem}
        Here the problem consists in considering that one should not fill only one backpack, but several which have a different size. An object cannot be in several backpacks. So we have to find the best combination so that the value of the objects in the backpacks added together is maximum, without exceeding the capacity of the backpacks.

\section{Brute-Force Approach}

    \textbf{Author :} Florian Doffemont

    \subsection{Definition}
        The Brute-Force method consists in trying all possible combinations of a problem. For example, to crack a password, we try all possibilities (a, b, ... aa, ab...).

        For the Knapsack problem, we will try all the possibilities of objects in the backpack.
    
    \subsection{Knapsack Problem}
    
        \subsubsection{Idea}
            First of all, we need an \emph{Items} structure that will define an object that can go in the backpack. The structure is defined as follows:

            \bigskip
            \begin{lstlisting}
class item:
    def __init__(self, pos, value, weight):
        self.pos = int(pos)
        self.value = float(value)
        self.weight = float(weight)
        self.ratio = float(value) / weight
            \end{lstlisting}
            \bigskip
            
            The structure has 4 fields, \emph{pos} which will indicate the position in a tree (useless in this problem), \emph{value} which will store the value of the object, \emph{weight} which will store the weight of the object and \emph{ratio} which will store the ratio value/weight of the object (useless here).\newline

            After that, we need a function to read the data sets (taken from the site: \emph{http://artemisa.unicauca.edu.co/~johnyortega/instances\_01\_KP/}).\newline

            With this function we will store in a list of \emph{Items} the \emph{Items} read and created. We will at the same time, store the maximum size of the backpack.\newline

            Now we just have to send the list of \emph{Items} and the maximum size of the backpack to the \emph{Brute-Force} algorithm.
            
        \subsubsection{Algorithm}
        
            The function is defined as follows:
            \begin{itemize}
                \item The function takes the list of items and the capacity of the knapsack as parameters.
                \item It returns the list of items that are in the knapsack with the optimal value
                \item It returns an empty list if no solution is found
            \end{itemize}
            
            So we have 2 parameters for the function :

            \bigskip
            \begin{lstlisting}
def bruteForce(capacity, items):
            \end{lstlisting}
            \bigskip
            
            At the beginning of the function, we will create 5 variables:
            \begin{itemize}
                \item \emph{start\_time} : Calculate the time of execution
                \item \emph{combinaisons} : Create a list of all possible combinations of items
                \item \emph{combinaisonsPossibles} : Create a list of all possible combinations of items that fit in the knapsack
                \item \emph{combinaisonsPossiblesMax} : Create a list of all possible combinations of items that fit in the knapsack and that have the highest value
                \item \emph{valeurMax} : We initialize the highest combination value to 0
            \end{itemize}

            \bigskip
            \begin{lstlisting}
start_time = time.time()
combinaisons = []
combinaisonsPossibles = []
combinaisonsPossiblesMax = []
valeurMax = 0
            \end{lstlisting}
            \bigskip
            
            Now you have to store all the possibilities (correct or not) in the \emph{combinaisons} list. We know that all possible combinations of items are $2^n$. We use the binary system to create all possible combinations of items. For example, if we have 3 items, we will have 8 possible combinations of items : 000, 001, 010, 011, 100, 101, 110, 111 ...\newline 
            000 means that we don't take any item, 001 means that we take the first item, 010 means that we take the second item, 011 means that we take the first and second items etc...

            \bigskip
            \begin{lstlisting}
for i in range(2**len(items)):
        combinaisons.append([])
        for j in range(len(items)):
            if (i >> j) % 2 == 1:
                combinaisons[i].append(items[j])
            \end{lstlisting}
            \bigskip
            
            With this list, we check if the combinations of items fit in the knapsack. If they fit, we add them to the list of possible combinations:

            \bigskip
            \begin{lstlisting}
for combinaison in combinaisons:
        poids = 0
        for item in combinaison:
            poids += item.weight
        if poids <= capacity:
            combinaisonsPossibles.append(combinaison)
            \end{lstlisting}
            \bigskip
            
            And finally, with this list containing the valid solutions, we check which combination of items has the highest value. If they have the same value, we add them to the list of possible combinations with the highest value. If they have a higher value, we empty the list of possible combinations with the highest value and we add the new combination. 
            \newpage
            If they have a lower value, we do nothing. We also update the highest value and the list of items that make up the highest value combination :

            \bigskip
            \begin{lstlisting}
for combinaison in combinaisonsPossibles:
        valeur = 0
        for item in combinaison:
            valeur += item.value
        if valeur > valeurMax:
            valeurMax = valeur
            combinaisonsPossiblesMax = [combinaison]
        elif valeur == valeurMax:
            combinaisonsPossiblesMax.append(combinaison)
            \end{lstlisting}
            \bigskip
            
            We obtain the list of items with the highest value and return it with the execution time :

            \bigskip
            \begin{lstlisting}
end_time = time.time()
execution_time = end_time - start_time
    
return combinaisonsPossiblesMax, execution_time
            \end{lstlisting}
            \bigskip
            
        \subsubsection{Analyze}

            The complexity is of $O(n*2^n)$ because we have as the "biggest" loop an $n$ loop in a $2^n$ loop, so $n*2^n$\newline\newline
            So it means that if we have 10 items, it will take $10,240$ operations to find the solution\newline
            If we have 20 items, it will take $20,971,520$ operations to find the solution\newline
            If we have 30 items, it will take $32,212,254,720$ operations to find the solution\newline
            So it is very slow, I think it is not recommended to use it with more than 20 items, but it always gives the optimal solution

    \newpage
    \subsection{Multidimensional Knapsack Problem}

        \subsubsection{Idea}
            First of all, like the simple Knapsack Problem, we need an \emph{ItemMD} structure that will define an object that can go in the backpack. The structure is defined as follows:

            \bigskip
            \begin{lstlisting}
class itemMD:
    def __init__(self,pos,value,weight, ratio):
        self.pos = pos # Position in the list of item
        self.value = value # Value of the item
        self.weight = weight # weight(vector) of the item for each dimension
        self.ratio = ratio # List of ratio for each dimension
            \end{lstlisting}
            \bigskip

            The structure has 4 fields, \emph{pos} which will indicate the position in a tree (useless in this problem), \emph{value} which will store the value of the object, \emph{weight} which will store the list of weight of the object for each dimension and \emph{ratio} which will store the ratio value/weight of the object for each dimensions (useless here).\newline

            Then we have to create a new function to read the data from the datasets, because it is not the same structure as for the Simple Knapsack. This function will return :

            \begin{itemize}
                \item The list of ItemMD read and created
                \item The number of dimension of the backpack
                \item The list of "max weight" of each dimension
                \item The optimal value to find, which is provided in the file
            \end{itemize}

            Now we just have to send the list of \emph{ItemMD}, the list of \emph{weight} for each dimensions and the number of \emph{dimension} to the \emph{Brute-Force-MD} algorithm.
            
        \newpage
        \subsubsection{Algorithm}

            The function is defined as follows:
                \begin{itemize}
                    \item It takes the list of itemMD, the list of capacities of the knapsacks and the number of dimensions as parameters.
                    \item It returns the best value, the list of items that can fit in all the dimensions of the knapsack and the time of execution
                    \item It returns an empty list if no solution is found
                \end{itemize}

            So we have 3 parameters for the function :

            \bigskip
            \begin{lstlisting}
def bruteForceMD(itemMD, capacities, nbrDimensions):
            \end{lstlisting}
            \bigskip

            At the beginning of the function, its the same than the Simple Knapsack, we will create 5 variables:
            \begin{itemize}
                \item \emph{start\_time} : Calculate the time of execution
                \item \emph{combinaisons} : Create a list of all possible combinations of items
                \item \emph{combinaisonsPossibles} : Create a list of all possible combinations of items that fit in the knapsack
                \item \emph{combinaisonsPossiblesMax} : Create a list of all possible combinations of items that fit in the knapsack and that have the highest value
                \item \emph{valeurMax} : We initialize the highest combination value to 0
            \end{itemize}

            \bigskip
            \begin{lstlisting}
start_time = time.time()
combinaisons = []
combinaisonsPossibles = []
combinaisonsPossiblesMax = []
valeurMax = 0
            \end{lstlisting}
            \bigskip

            \newpage
            Now we have to store all the possibilities, we use the same method as for the Simple Knapsack :

            \bigskip
            \begin{lstlisting}
for i in range(2**len(items)):
        combinaisons.append([])
        for j in range(len(items)):
            if (i >> j) % 2 == 1:
                combinaisons[i].append(items[j])
            \end{lstlisting}
            \bigskip

            This is where the algorithm will change compared to the previous one, we will put in the \emph{combinaisonsPossibles} list all the possibilities that fit in the backpack. Contrary to the previous algorithm that checked if the combination of objects fit in the backpack, here we will check that the list of objects fit not only in one dimension, but in the $d$ dimensions:

            \bigskip
            \begin{lstlisting}
for combinaison in combinaisons:
        poids = [0] * nbrDimensions
        for item in combinaison:
            for i in range(nbrDimensions):
                poids[i] += item.weight[i]
        if poids <= capacities:
            combinaisonsPossibles.append(combinaison)
            \end{lstlisting}
            \bigskip

            And finally, with this list containing the valid solutions, we will do as for the previous algorithm, keep only the combination(s) that have the highest value : 

            \bigskip
            \begin{lstlisting}
for combinaison in combinaisonsPossibles:
        valeur = 0
        for item in combinaison:
            valeur += item.value
        if valeur > valeurMax:
            valeurMax = valeur
            combinaisonsPossiblesMax = [combinaison]
        elif valeur == valeurMax:
            combinaisonsPossiblesMax.append(combinaison)
            \end{lstlisting}
            \bigskip

            We obtain the list of items with the highest value and return it with the execution time and the optimal value found :

            \bigskip
            \begin{lstlisting}
end_time = time.time()
execution_time = end_time - start_time

return combinaisonsPossiblesMax, execution_time, valeurMax
            \end{lstlisting}
            \bigskip

        \subsubsection{Analyze}

            The complexity is of $O(n*2^n)$ because we have as the "biggest" loop an $n$ loop in a $2^n$ loop, so $n*2^n$\newline\newline
            So it means that if we have 10 items, it will take $10,240$ operations to find the solution\newline
            If we have 20 items, it will take $20,971,520$ operations to find the solution\newline
            If we have 30 items, it will take $32,212,254,720$ operations to find the solution\newline
            So it is very slow, I think it is not recommended to use it with more than 20 items, but it always gives the optimal solution

\section{Branch and Bound Approach}

    \textbf{Author :} Jeoffrey Pereira

    \subsection{Definition}
    
    \subsection{Knapsack Problem}

\section{Three Greedy Approach}

    \textbf{Author :} Jocelyn Hauf

    \subsection{Definition}
    
    \subsection{Knapsack Problem - Greedy 1}
    
    \subsection{Knapsack Problem - Greedy 2}
    
    \subsection{Knapsack Problem - Greedy 3}

\section{Dynamic Programming Approach}

    \textbf{Author :} Arthur Micol

    \subsection{Definition}
    
    \subsection{Knapsack Problem}

\section{Fully Polynomial Time Approximation Scheme Approach}

    \textbf{Author :} 

    \subsection{Definition}
    
    \subsection{Knapsack Problem}

\section{Randomized Approach}

    \textbf{Author :}

    \subsection{Definition}
    
    \subsection{Knapsack Problem}

\section{Ant Colony Approach}

    \textbf{Author :}

    \subsection{Definition}
    
    \subsection{Knapsack Problem}

\section{Personal Approach}

    \textbf{Author :}

    \subsection{Definition}
    
    \subsection{Knapsack Problem}

\section{Conclusion}

\end{document}



